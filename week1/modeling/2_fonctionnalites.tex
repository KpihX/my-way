\section{Fonctionnalités d'une application pouvant transformer le secteur du transport urbain et inter-urbain}

\begin{enumerate}
\item \textbf{Calcul d'itinéraire intelligent :}
L’application devrait permettre aux utilisateurs de saisir leur point de départ et leur destination, puis générer une liste d'itinéraires classés par ordre d'optimalité en tenant compte des modes de transport disponibles (bus, taxis, moto-taxis, etc.). L'application pourrait aussi proposer le calcul d'itinéraire par filtre (distance, coût, surcharge de la route, état de la route, etc.).
\item \textbf{Calcul d'itinéraire en temps réel :}
Le calcul d'itinéraire optimal pour l'utilisateur devrait pouvoir se faire automatiquement en temps réel lorsqu'il se déplace par rapport à son point d'arrivée.

\item \textbf{Cartographie détaillée :}
- Afficher des cartes interactives avec des points d’intérêt, des arrêts de transport, des stations-service, etc.
- Proposer des vues 3D pour faciliter la navigation.

\item \textbf{Informations en temps réel :}
Fournir des mises à jour en temps réel sur les conditions de circulation, les retards et les incidents sur les routes pour permettre aux utilisateurs de planifier leurs déplacements en conséquence, et même de changer d'itinéraire en cas de problème.

\item \textbf{Indications précises :}
Fournir des indications précises et détaillées, y compris les points d'intérêt, les intersections, les changements de direction, etc., pour aider les utilisateurs à naviguer facilement dans la ville.

\item \textbf{Navigation hors ligne :}
Permettre aux utilisateurs de télécharger des cartes et des itinéraires pour une navigation hors ligne dans les zones où la connectivité Internet est limitée.

\item \textbf{Collaboration avec d'autres services :}
L'application devrait permettre de collaborer avec d'autres services tels que la collecte des clients sur un itinéraire, le suivi d'un voyage, et autres.

\item \textbf{Statistiques de déplacement et IA :}
L'application pourrait enregistrer les différentes données statistiques de déplacement de ses utilisateurs qui pourraient aider à des prises de décision et même aux prédictions sur le temps de parcours, la surcharge de certaines routes, et proposer un modèle d'intelligence artificielle pour différentes prévisions et conseils.

\item \textbf{Suivi personnalisé et Synchronisation avec le calendrier :}
L'application devrait pouvoir utiliser les données et préférences de déplacement d'un utilisateur pour lui proposer un service client personnalisé automatique et lui offrir des conseils ; cette partie peut également être basée sur l'IA. L’application devrait également utiliser le calendrier des utilisateurs pour gérer les itinéraires de voyage en fonction de leur emploi du temps.

\item \textbf{Proposition de réservations :}
Afin d'éviter les désagréments souvent causés par des routes bloquées par le gouvernement lors des journées nationales ou lors du déplacement du chef de l'État, l'application devrait pouvoir permettre au gouvernement de réserver des routes à cet effet afin d'éviter de pénaliser les usagers et d'exploiter au mieux les routes libres dans ces situations.

\item \textbf{Partage d’itinéraire :}
Permettre aux utilisateurs de partager facilement leur itinéraire de voyage avec des amis ou des membres de la famille.

\item \textbf{Intégration des avis :}
L'application devrait incorporer des avis et des recommandations d'autres voyageurs sur des étapes spécifiques de l'itinéraire.
\end{enumerate}

