\documentclass[12px]{article}
\usepackage[utf8]{inputenc}
\usepackage{lipsum} % Package used to generate dummy text
\usepackage{tocloft} % Package used to format table of contents
\usepackage{titlesec} % Package used to format titles
\usepackage{geometry} % Package used to adjust page layout

% Formatting titles
\titleformat{\section}
  {\normalfont\Large\bfseries}{\thesection}{1em}{}
\titleformat{\subsection}
  {\normalfont\large\bfseries}{\thesubsection}{1em}{}
\titleformat{\subsubsection}
  {\normalfont\normalsize\bfseries}{\thesubsubsection}{1em}{}

% Formatting table of contents
\renewcommand{\cftsecfont}{\normalfont}
\renewcommand{\cftsecpagefont}{\normalfont}
\renewcommand{\cftsubsecfont}{\normalfont}
\renewcommand{\cftsubsecpagefont}{\normalfont}
\renewcommand{\cftsubsubsecfont}{\normalfont}
\renewcommand{\cftsubsubsecpagefont}{\normalfont}

% Adjusting page layout
\geometry{margin=1in} % Set 1-inch margins

\title{Rapport de Réunion de Groupe}
\author{DJONGO Ariel\and DONCHI Tresor \and KAMDEM Ivann \and NOMO Gabriel}
\date{\today}

\begin{document}

\begin{titlepage}
    \centering
    \vspace*{1cm}
    \Large\textbf{Cours de Réseaux Informatiques}\par
    \vspace{0.5cm}
    \Large\textbf{Projet : My-Way}\par
    \vspace{0.5cm}
    \Large\textbf{\textit{Un outil efficace pour la gestion et le parcours d'itinéraires}}\par
    \vspace{1cm}
    \vspace{1cm}
    \Large\textbf{Rapport de Quatrième Réunion de Groupe}\par
    \vspace{1cm}
    \textit{Réalisé par les étudiants:\\}
    \textbf{DJONGO Ariel}\par
    \textbf{DONCHI Tresor}\par
    \textbf{KAMDEM Ivann}\par
    \textbf{NOMO Gabriel}\par
    \vspace{0.6cm}\par
    \textbf{3 GI\\}
    \vspace{0.4cm}
    \textit{Sous la supervision de:\\}
    \textbf{Pr Thomas DJOTIO}\par
    \textbf{Mr Juslin KUTCHE}\par
    \vspace{6cm}
    \today\par
    \vfill
\end{titlepage}

\tableofcontents
\clearpage


\section{Introduction}
Dans le but de respecter le processus d'évolution fixé à l'entame du projet, une réunion de mise au point s'est tenue le \textit{Vendredi, 05 Avril 2024} dans la salle B-34B au BP1 de l'ENSPY.
Le bilan de cette réunion est présenté dans ce document.

\newpage
\section{Circonstances}
\begin{itemize}
    \item \textbf{Heure de début fixée:} \textit{13:00}
    \item \textbf{Heure de début effective:} \textit{13:10}
    \item \textbf{Présence:} \textit{Tous les quatre membres étaient présents.}
    \item \textbf{Heure de fin:} \textit{16:32}
\end{itemize}
%--------------------------///  ////-----------------------------------
\section{Première partie: Concertation des membres}
Au tout début de notre réunion, nous avons premièrement rappelés les objectifs de celle-ci.
Puis, nous avons commencé par le premier sujet: L'initiation à la communication avec l'API OpenStreetMap. L'opération a été plus ou moins bien accomplie, mais il en est ressorti que nous devrions y faire plus de recherches.
Le second sujet était la finalisation de notre modèle de classes. Il s'en est suivi les modifications suivantes:
\begin{itemize}
    \item \textbf{Modèle de contexte:}
    \begin{itemize}
        \item \textit{Retrait de usager et conducteur (car ses deux clients avaient trop peu de points de dissemblance) et ajout de Client}
    \end{itemize}
    \item \textbf{Modèle de classes:}
    \begin{itemize}
        \item Classe \textbf{\textit{Usager}}: \textbf{\textit{Supprimée}} (La similarité avec Conducteur dans notre modélisation était trop forte)
        \item Ajout de la classe \textbf{\textit{Particulier}}: personne capable de réserver une route
        \item Création des interfaces \textbf{\textit{Displayable}} pour les objets "\textit{affichables sur les cartes}" et \textbf{\textit{CostEvaluation}}
        \item Création de l'interface \textbf{\textit{GraphUtils}} qui ne contiendra que pour un début les algorithmes de parcours des graphes.
    \end{itemize}
\end{itemize}

C'est après s'être mis d'accord sur ces modifications, nous avons décidé de nous rendre dans le bureau de l'enseignant: il était alors 16:02.

\section{Deuxième partie: Rencontre avec le Pr DJOTIO}
Pendant et après la présentation de notre modélisation au Pr DJOTIO, dans son bureau, il nous a fourni bon nombre de remarques et commentaires.
\begin{itemize}
    \item \textbf{Modèle de contexte}
    \begin{itemize}
        \item Ajouter Syndicat comme acteur en tant que celui qui autorise ou non un utilisateur à utiliser l'application (acteur secondaire) et retirer plate-forme
        \item Discriminer les types d'utilisateurs:(conducteur ou non, véhiculé ou non)
        \begin{itemize}
            \item exemple sans véhicule: personnel, client lambda....
            \item se concentrer sur les chauffeurs indépendant (s'enregistrer, définir ses itinéraires)
        \end{itemize}
    \end{itemize}
    \item \textbf{Modèle de package}
    \begin{itemize}
        \item Ajout du module de gestion des identités
    \end{itemize}
    \item \textbf{Modèle de classe}
    \begin{itemize}
        \item Enregistrer les mots-clés et photo de profil pour un user
    \end{itemize}
    \item \textbf{Autres}
    \begin{itemize}
        \item Statistiques générales de trafic sur les itinéraires à Yaoundé.
        \item Il est à noter que nous avons demander à l'enseignant les codes sources pour la communication avec l'API d'OSM, mais il n'a pas pu nous le fournir. Cependant, il a assuré qu'il les cherchera et nous les fournira dès lors.
    \end{itemize}
\end{itemize}
\section{Planification de la prochaine réunion:}
Elle est prévue pour le \textit{Dimanche, 07 Avril 2024} dès \textit{08h00} par un \textit{meet Google.}
Les tâches à réaliser avant cette échéance sont réparties comme suit:
\begin{itemize}
    \item \textbf{\textit{Un cahier d'analyse et de conception}} à la charge de \textit{KAMDEM} et \textit{NOMO}.
    \item \textit{\textbf{Un cahier de charges}} à la charge de \textit{DJONGO} et \textit{DONCHI}.
    \item Les statistiques 
\end{itemize}
Nous devons également faire davantage de recherches sur la communication avec l'API JS d'OpenStreetMap.
Au cours de cette réunion, nous réviserons les réalisations de chacun et planifierons une éventuelle évaluation de notre projet lundi.
\newpage

\section{Conclusion}
En définitive, le bilan de cette réunion est positif. Elle nous a permis de nous poser sur l'évolution globale de notre projet et à prévoir les actions futures. La prochaine réunion a été fixée d'un commun accord. Nous espérons que nous respecterons à la lettre cet engagement..

\end{document}
%--------------------------///  ////-----------------------------------