% Introduction
\section{Introduction}
Ce document vise à modéliser le problème de gestion et de parcours d'itinéraires dans une ville donnée. Nous entamons cette démarche en procédant à une analyse statistique détaillée afin de mieux appréhender les défis et les enjeux associés à la mobilité urbaine. Ensuite, nous explorerons les différentes fonctionnalités potentielles nécessaires pour résoudre ce problème, en tenant compte des besoins des utilisateurs et des contraintes technologiques. Nous aborderons également l'importance du sous-secteur dans la transformation digitale du pays, mettant en lumière les avantages économiques, sociaux et environnementaux que pourrait apporter une solution innovante. Par la suite, nous examinerons les scénarios d'utilisation possibles, les acteurs impliqués et les implications sur l'environnement technique nécessaire au déploiement de l'application. Enfin, nous proposerons une modélisation mathématique du problème à l'aide de la théorie des graphes, offrant ainsi une approche structurée et rigoureuse pour résoudre efficacement les défis de gestion et de parcours d'itinéraires.

