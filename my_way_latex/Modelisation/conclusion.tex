% Conclusion
\section{Conclusion}
En conclusion, ce cahier de modélisation nous a permis d'explorer en profondeur le problème de gestion et de parcours d'itinéraires dans une ville donnée. À travers une analyse approfondie, nous avons identifié les principaux défis et enjeux associés à la mobilité urbaine, et nous avons proposé des solutions potentielles pour y faire face. L'importance du sous-secteur dans la transformation digitale du pays a été soulignée, mettant en évidence les nombreuses opportunités qu'offre une approche innovante pour améliorer la qualité de vie des citoyens et stimuler le développement économique. En adoptant une approche méthodique et rigoureuse, nous avons développé une modélisation mathématique du problème, offrant ainsi une base solide pour la conception et la mise en œuvre d'une solution efficace et durable. Il est maintenant temps de passer à l'étape suivante : l'implémentation de cette solution pour créer un impact positif sur la mobilité urbaine et contribuer au développement global de notre société.
