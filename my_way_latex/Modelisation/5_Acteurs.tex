\section{Acteurs}
Les principaux acteurs impliaués dans la réalisation t le déploiement de notre solution sont:
\begin{itemize}
    \item \textbf{Le gouvernement:} Le gouvernement camerounais a mis en place une stratégie nationale de développement de l'économie numérique qui inclut la digitalisation des services de transport.
    \item \textbf{Les opérateurs de télécommunication:}  Les opérateurs de télécommunications investissent dans le développement de l'infrastructure internet et des services digitaux au Cameroun.
    \item \textbf{Les startup:} De nombreuses startups camerounaises développent des solutions digitales pour la mobilité urbaine. (Cas de Yango)
    \item \textbf{Les entreprises:} Les Grandes et Très Grandes Entreprises de renom comme des hôtels de luxe ou des centres commerciaux ont besoin de pouvoir fournir aisément leur localisation.
    \item  \textbf{Les services de transports} (tels que les taxis et mototaxis)
\end{itemize}

Le développement d'une application de gestion d'itinéraires à Yaoundé nécessite une collaboration entre les différents acteurs du secteur de la mobilité urbaine et de la transformation digitale. En tirant parti des synergies et des collaborations, il est possible de créer une solution durable et inclusive qui répond aux besoins des citoyens camerounais

En ce qui concerne le cas particulier de l'itinraire, il est le fil conducteur qui relie tous les aspects du voyage, de la planification à l’exécution.
\begin{itemize}
    \item \textbf{Voyage urbain et interurbain :} L’itinéraire détermine le choix des modes de transport et les trajets, influençant ainsi les décisions en matière de voyage urbain et interurbain.
    \item \textbf{Conducteurs indépendants :} Que ce soit avec ou sans voiture, les conducteurs s’appuient sur des itinéraires optimisés pour offrir des services efficaces et répondre aux besoins des voyageurs.
    \item \textbf{Voyageurs indépendants :} Les itinéraires personnalisés aident les voyageurs sans voiture à naviguer dans les systèmes de transport urbain et interurbain, tandis que ceux avec voiture bénéficient de la planification de trajets.
    \item \textbf{Agences de voyage :} Elles utilisent les itinéraires pour créer des offres de voyage attrayantes et pour établir des connexions avec d’autres agences et prestataires de services.
    \item \textbf{Agences de location de véhicules :} Les itinéraires influencent la disponibilité et la gestion des flottes de véhicules, ainsi que les services offerts aux clients.
    \item \textbf{Planification de voyage :} La planification d’itinéraires est essentielle pour la coordination des ressources telles que les véhicules et les conducteurs, et pour la publication de voyages.
    \item \textbf{Réservation de voyage :} Les itinéraires sont au cœur des systèmes de réservation, permettant aux voyageurs de rechercher, confirmer, payer et évaluer leurs voyages.
    \item \textbf{Intermédiaires du secteur de voyage :} Ils facilitent la correspondance entre les itinéraires des voyageurs et les services disponibles, améliorant ainsi l’efficacité des réservations.
    \item \textbf{Activités à valeur ajoutée :} Les itinéraires aident à intégrer des services tels que le tourisme, l’hébergement, la restauration et les événements, enrichissant l’expérience globale du voyage.
    \item \textbf{Sécurité des acteurs pendant le voyage :} Les itinéraires bien planifiés contribuent à la sécurité des voyageurs en optimisant les trajets et en réduisant les risques.
\end{itemize}