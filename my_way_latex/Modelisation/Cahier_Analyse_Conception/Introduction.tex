% Introduction
\section{Introduction}

Dans le cadre du développement des infrastructures de transport en milieu interurbain, la mise en place d'une plateforme de gestion des itinéraires se présente comme une solution innovante pour répondre aux défis de mobilité urbaine. Ce projet vise à concevoir une application capable de faciliter la planification des voyages et d'optimiser les parcours au sein de la ville de Yaoundé.

La ville de Yaoundé, caractérisée par son dynamisme et sa croissance démographique, se trouve confrontée à des enjeux majeurs en termes de gestion de trafic et de planification des déplacements. L'objectif de ce système est de proposer des itinéraires optimaux, en tenant compte des divers facteurs tels que le temps de parcours, la distance, et la densité de trafic, afin d'améliorer l'expérience des usagers et de contribuer à la fluidité du transport interurbain.

Le présent cahier d'analyse et de conception documentera de manière exhaustive les besoins, les exigences, et les spécifications techniques du système. Il servira de référence tout au long du cycle de vie du projet, depuis la phase d'analyse préliminaire jusqu'à la validation finale du produit. Ce projet, tout en se concentrant initialement sur la gestion des itinéraires dans la ville de Yaoundé, pourra éventuellement s'intégrer dans un projet plus grand de gestion des voyages en milieu interurbain.


